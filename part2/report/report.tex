\documentclass[11pt,a4paper]{article}
\usepackage{graphicx}
\usepackage{amsmath}
\usepackage{booktabs}
\usepackage{geometry}
\usepackage{caption}
\usepackage{subcaption}
\usepackage{hyperref}
\geometry{margin=1in}

\title{\textbf{Assignment 4 - Part 2: Congestion Control Implementation and Analysis}}
\author{Name: Arpit Prasad and Akshat Bhasin \\ Roll No: 2022EE11837 and 2022EE3 \\ Course: Computer Networks}
\date{}

\begin{document}
\maketitle

\section{Fixed Bandwidth Experiment}

\textbf{Observation:} From the plot, we observe that the Jain’s Fairness Index (JFI) remains constant and equal to 1 across all link capacities, indicating perfect fairness among flows. However, the link utilization decreases sharply as link capacity increases.

\textbf{Reason:} The constant JFI = 1, shows the CCA correctly maintains the fairness of in the given topology. Howver the dropping link utilisation is an effect of the having a fixed queue size on the switches, which limits the max throughput with which files may be transferred from the server to the client. Hence, as bandwidth increases, the link utilisation decreases.

\begin{figure}[h!]
  \centering
  \includegraphics[width=0.6\textwidth]{/home/arpit/Desktop/iitd/sem_7/COL334/projects/Reliable-UDP-Wrapper/part2/p2_fairness_fixed_bandwidth.png}
  \caption{Link utilization and JFI vs link capacity }
\end{figure}

\section{Varying Loss Experiment}

\textbf{Observation:} Link utilization decreases significantly with increasing packet loss rate. The curve shows a steep decline as loss increases from 0\% to 2\%.

\textbf{Reason:} As the loss rate increases, the reliability algorithm resends packets, hence taking larger time to deliver the a single packet. This leads to reduced throughput and hence reduced link utilisation

\begin{figure}[h!]
  \centering
  \includegraphics[width=0.6\textwidth]{/home/arpit/Desktop/iitd/sem_7/COL334/projects/Reliable-UDP-Wrapper/part2/p2_fairness_varying_loss.png}
  \caption{Effect of loss on link utilization }
\end{figure}

\section{Asymmetric Flows Experiment}

\textbf{Observation:} The JFI remains constant and equal to 1 even when one flow experiences increasing RTT delay. This shows that the system maintains fairness regardless of delay asymmetry.

\textbf{Reason:} Since TCP Cubic ideas are implemented, it is made RTT independent and hence does not produce a bias towards smaller RTT connections. Hence JFI is close to 1.

\begin{figure}[h!]
  \centering
  \includegraphics[width=0.6\textwidth]{/home/arpit/Desktop/iitd/sem_7/COL334/projects/Reliable-UDP-Wrapper/part2/p2_fairness_asymmetric_flows.png}
  \caption{Fairness (JFI) vs RTT difference }
\end{figure}

\section{Background UDP Traffic}

\textbf{Observation:} As the background UDP load increases (from Light to Heavy), link utilization decreases slightly, while JFI remains nearly 1. This indicates fairness is preserved even under heavy interference.

\textbf{Reason:} Since the queue size on the switches are fixed, therefore when another bursty flow is introduced it keeps the high percentage of the queue occupied. This causes the loss of packets sent based on reliability, which leads to retransmission, hence decrease in throughput and link utilisation.

\begin{figure}[h!]
  \centering
  \includegraphics[width=0.6\textwidth]{/home/arpit/Desktop/iitd/sem_7/COL334/projects/Reliable-UDP-Wrapper/part2/p2_fairness_background_udp.png}
  \caption{Impact of background UDP traffic on utilization and fairness }
\end{figure}

\end{document}
