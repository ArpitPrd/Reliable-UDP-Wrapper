\documentclass[11pt,a4paper]{article}
\usepackage{graphicx}
\usepackage{amsmath}
\usepackage{booktabs}
\usepackage{geometry}
\usepackage{caption}
\usepackage{subcaption}
\usepackage{hyperref}
\geometry{margin=1in}

\title{\textbf{Assignment 4 - Part 2: Congestion Control Implementation and Analysis}}
\author{Name: Arpit Prasad and Akshat Bhasin \\ Roll No: 2022EE11837 and 2022EE3 \\ Course: Computer Networks}
\date{}

\begin{document}
\maketitle

\section*{1. Introduction}
This report describes the implementation and evaluation of a congestion control algorithm (CCA) built over the reliable UDP layer from Part 1. The CCA aims to maximize link utilization while maintaining fairness between flows.

\section*{2. Algorithm Design}
\subsection*{2.1 Congestion Window Management}
\begin{itemize}
  \item \textbf{Slow Start:} cwnd starts at 1 MSS and doubles each RTT until \texttt{ssthresh}.
  \item \textbf{Congestion Avoidance:} Additive increase once threshold is reached.
  \item \textbf{Loss Handling:} Triple duplicate ACK → halve cwnd; timeout → reset to 1 MSS.
\end{itemize}

\subsection*{2.2 Implementation Details}
All reliability mechanisms from Part 1 are preserved. Additional logging tracks \texttt{cwnd}, \texttt{ssthresh}, and ACK events for analysis.

\section*{3. Experimental Setup}
Experiments use a dumbbell topology with two client–server pairs sharing a bottleneck link. Traffic conditions such as bandwidth, loss, and delay are varied.

\section*{4. Results and Analysis}

\subsection*{4.1 Fixed Bandwidth Experiment}
\begin{figure}[h!]
  \centering
  \fbox{\parbox[c][6cm][c]{0.8\linewidth}{\centering Placeholder for Link Utilization and JFI vs Link Capacity plot}}
  \caption{Link utilization and JFI vs link capacity (placeholder).}
\end{figure}

\subsection*{4.2 Varying Loss Experiment}
\begin{figure}[h!]
  \centering
  \fbox{\parbox[c][6cm][c]{0.8\linewidth}{\centering Placeholder for Loss Rate vs Link Utilization plot}}
  \caption{Effect of loss on link utilization (placeholder).}
\end{figure}

\subsection*{4.3 Asymmetric Flows Experiment}
\begin{figure}[h!]
  \centering
  \fbox{\parbox[c][6cm][c]{0.8\linewidth}{\centering Placeholder for RTT vs JFI plot}}
  \caption{Fairness (JFI) vs RTT difference (placeholder).}
\end{figure}

\subsection*{4.4 Background UDP Traffic}
\begin{figure}[h!]
  \centering
  \fbox{\parbox[c][6cm][c]{0.8\linewidth}{\centering Placeholder for Background UDP Impact plot}}
  \caption{Impact of background UDP traffic on utilization and fairness (placeholder).}
\end{figure}

\section*{5. Observations}
\begin{itemize}
  \item The CCA achieves high utilization and maintains fairness across flows.
  \item Increasing loss reduces throughput as expected due to window reduction.
  \item Background traffic impacts JFI but utilization remains stable for light load.
\end{itemize}

\section*{6. Conclusion}
The implemented congestion control algorithm adapts effectively to bandwidth, loss, and delay variations, achieving balanced efficiency and fairness.

\end{document}
