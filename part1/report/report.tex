\documentclass[11pt,a4paper]{article}
\usepackage{graphicx}
\usepackage{caption}
\usepackage{subcaption}
\usepackage{amsmath}
\usepackage{geometry}
\usepackage{hyperref}
\usepackage{booktabs}
\geometry{margin=1in}

\title{\textbf{Assignment 4 – Part 1: Reliable Data Transfer over UDP}}
\author{Name: \underline{\hspace{4cm}} \\ Roll No: \underline{\hspace{3cm}} \\ Course: Computer Networks}
\date{}

\begin{document}
\maketitle

\section*{1. Introduction}
This report presents the design and analysis of a reliable file transfer protocol implemented over UDP. Since UDP is unreliable, reliability mechanisms were implemented at the application layer using a sliding window approach.

\section*{2. Design Overview}
\subsection*{2.1 Header Structure}
Each packet has a maximum payload size of 1200 bytes with a 20-byte custom header:
\begin{center}
\texttt{| Seq Num (4 B) | Reserved/Optional (16 B) | Data (($\leq$ 1180 B) |}
\end{center}

\subsection*{2.2 Reliability Mechanisms}
\begin{itemize}
  \item \textbf{Sliding Window:} Ensures in-order delivery and controlled transmission.
  \item \textbf{Acknowledgments:} Cumulative ACKs; optional SACK support.
  \item \textbf{Timeouts and Retransmissions:} RTO is estimated adaptively.
  \item \textbf{Fast Retransmit:} Triggered after three duplicate ACKs.
\end{itemize}

\section*{3. Experimental Setup}
Experiments were run in Mininet using a two-host topology (\texttt{h1–s1–h2}) with variable delay, jitter, and loss.

\section*{4. Results and Analysis}

\subsection*{4.1 Download Time vs Loss Rate}
\begin{figure}[h!]
  \centering
  \fbox{\parbox[c][6cm][c]{0.8\linewidth}{\centering Placeholder for Download Time vs Loss Rate plot}}
  \caption{Download time vs packet loss rate (placeholder).}
\end{figure}

\subsection*{4.2 Download Time vs Delay Jitter}
\begin{figure}[h!]
  \centering
  \fbox{\parbox[c][6cm][c]{0.8\linewidth}{\centering Placeholder for Download Time vs Delay Jitter plot}}
  \caption{Download time vs delay jitter (placeholder).}
\end{figure}

\section*{5. Observations}
\begin{itemize}
  \item Download time increases with packet loss due to retransmissions.
  \item Jitter affects timing but less severely than loss.
  \item Fast retransmit reduces latency compared to timeout-based recovery.
\end{itemize}

\section*{6. Conclusion}
The protocol ensures in-order and complete file delivery under adverse conditions, achieving reliability similar to TCP at the application layer.

\end{document}
