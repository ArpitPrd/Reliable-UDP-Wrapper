\documentclass[11pt,a4paper]{article}
\usepackage{graphicx}
\usepackage{caption}
\usepackage{subcaption}
\usepackage{amsmath}
\usepackage{geometry}
\usepackage{hyperref}
\usepackage{booktabs}
\geometry{margin=1in}

\title{\textbf{Assignment 4 – Part 1: Reliable Data Transfer over UDP}}
\author{Name: Arpit Prasad and Akshat Bhasin \\ Roll No: 2022EE11837 and 2022EE31996 \\ Course: Computer Networks}
\date{}

\begin{document}
\maketitle

\section{Header Structure}
When server sends to client we have: (here we do not use the reserved bytes section)

\texttt{| 0-3 : Sequence Number | 4-19 : Reserved (all zeros) | 20-... : Data |}

When client sends to server: we use the reserved bits for Selective ACK, implementation where we support selectively ack-ing upto two disjoint set of seq number ranges. This increases the throughput.

\texttt{| 0-3 : ACK Number | 4-7 : SACK1 Start | 8-11 : SACK1 End | 12-15 : SACK2 Start | 16-19 : SACK2 End |}


\section{Download Time vs Loss Rate}
\textbf{Observation}: As the loss rate increases, the download time increases

\textbf{Reason}: As loss rate increases the number of packets lost in flight increases. Hence the reliability mechnaism either through RTOs or Duplicate ACKs ends up retransmitting the packets. Therefore the same amount of data requires more time to transfer than non-lossy links, this increases the time of successfull reception of packets. This affect cascades to the subsequent packets and hence delays the complete file transfer.

\begin{figure}[h!]
  \centering
  \includegraphics[width=0.8\textwidth]{/home/arpit/Desktop/iitd/sem_7/COL334/projects/Reliable-UDP-Wrapper/part1/plot_loss_experiment.png}
  \caption{Download time vs packet loss rate}
\end{figure}

\section{Download Time vs Delay Jitter}

\textbf{Observation}: As jitter increases the download time increases.

\textbf{Reason}: As jitter increases, the variation in RTT increases. This leads to instability in RTO estimation. The sender may either retransmit too early or too late, increasing total download time

\begin{figure}[h!]
  \centering
  \includegraphics[width=0.8\textwidth]{/home/arpit/Desktop/iitd/sem_7/COL334/projects/Reliable-UDP-Wrapper/part1/plot_jitter_experiment.png}
  \caption{Download time vs delay jitter}
\end{figure}

\end{document}
